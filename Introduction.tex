\section{Motivations}

The exascale era of HPC will come soon, and the first real exascale machine will be available around 2020, in which, the researchers hope to make unprecedented advancements in many fields of sciences and industries. In fact, the most powerful machines now have already over a million cores, and the HPC cluster systems will continue not only to scale up in compute node and Central Processing Unit (CPU) core count, but also the increase of components heterogeneity with the introduction of the Graphics Processing Unit (GPU) and other accelerators. This trend causes the transition to multi- and many cores inside of computing nodes which communicate explicitly through faster interconnection networks. These hierarchical supercomputers can be seen as the combination of distributed and parallel computing. Nevertheless, only a small number of applications may attain sustainable performances due to their lack of good scalability on large clusters. 

Many scientific and industrial applications can be formlated as linear systems. A linear system can be described as a operation $A$, a input $x$ and a output $b$. The linear solvers which aim to solve these systems, are the kernel of most simulation applications and softwares. When the operation matrix $A$ is sparse, the collection of Krylov subspace methods, such as the Generalized minimal residual method (GMRES), the Conjugate Gradient (CG)  and the Biconjugate Gradient Stabilized Method (BiCGSTAB), are well-known tools to solve the linear systems. Krylov subspace methods can approximate the exact solution of linear systems inside the Krylov subspace starting from a given initial guess vector. 

Linear systems describe the real applications, such as the fusion operations, the earthquake and weather forecasting, etc., and their dimensions grow quickly (e.g. more than $10$ billions unknows for the earthquake simulation) with the increase of the complexity of applications. Krylov subspace methods should be deployed on the supercomputing platforms to solve such large-scale linear systems. Nowadays, with the increase of computing unit number and the heterogeneity of supercomputers, the communication of overall reduction and global synchronization of Krylov subspace methods are a bottleneck, which damages heavily their parallel performance. In details, when solving a large-scale problem on parallel architectures by Krylov subspace methods, the cost per iteration of the method becomes the most significant concern, typically because of communication and synchronization overhead. Consequently, large scalar products, overall synchronization, and other operations involving communication among all cores have to be optimized. The numerical applications should be optimized for more local communication and less global communication. To benefit the full computational power of such hierarchical systems, it is central to explore novel parallel methods and models for the solving of linear systems. These methods should not only accelerate the convergence but also have the abilities to adapt to multi-grain, multi-level memory, to improve the fault tolerance, reduce synchronization and promote asynchronization.

\section{Objectives and Core Contributions}

The subject of this dissertation fits within this research context and it concerns on the propose and analyze a distributed and parallel programming paradigm for smart hybrid Krylov methods targetting at the exascale computing. The research relies on the Unite and Conquer approach proposed by Emad \cite{emad2016unite}. This approach is a model for the design of numerical methods by combining different computation components together to work for the same objective, with asynchronous communication among them. Unite implies the combination of different calculation components, and conquer represents different components work together to solve one problem. In the unite and conquer methods, different computation parallel components work independently with asynchronous communication. These different components can be deployed on different platforms such as P2P, cloud and the supercomputer systems, or on the same platform with different processors. The idea of unite and conquer approach came from the article of Saad \cite{saad1984chebyshev} where he suggested using Chebyshev polynomial to accelerate the convergence of Explicitly Restarted Arnoldi Method (ERAM) to solve eigenvalue problems. The ingredients of this thesis to construct a Unite and Conquer linear solver come from the previous research of hybrid methods, e.g. a hybrid Chebyshev Krylov subspace algorithm proposed by Elman \cite{elman1986hybrid}, a hybrid GMRES algorithm via a Rchardson iteration with Leja ordering \cite{nachtigal1992hybrid}, an approach for solving a system of linear equations which takes a combination of two arbitrary approximate solutions of two methods introduced by Brezinski \cite{brezinski1994hybrid}, and a hybrid gmres/ls-arnoldi method firstly introduced by Essai \cite{essai1999heterogeneous}.

Before the investigation on the Krylov subspace methods to solve linear systems, the first contribution of this work is to develop a Scalable Matrix Generator with Given Spectrum (SMG2S), due to the importance of spectral distribution on the convergence of iterative methods. SMG2S allows generating very large dimension non-Hermitian matrices with user-customized eigenvalues. These matrices are also non-Hermitian and non-trivial, with very high dimension. In fact, recent research related to social networking, big data, machine learning and artificial intelligence has increased the necessity for non-hermitian solvers associated with much larger sparse matrices and graphs. Iterative linear algebra methods are the important parts of the overall computing time of applications in various fields since decades. The analysis of the iterative method behaviors for such problems is complex, and it is necessary to evaluate their convergence to solve extremely large non-Hermitian eigenvalue and linear problems on parallel and/or distributed machines. Since the convergence of iterative methods depends on the properties of spectra, it is necessary to generate large matrices with known spectra to benchmark them. The motivation to propose SMG2S is that it does not exist a collection of test matrices with very large dimension and different kinds of spectral properties to benchmark the linear solvers on supercomputers. SMG2S serves as a very useful tool during my thesis to study the numerical and parallel performance of new designed iterative methods.

After the implementation of SMG2S, the second contribution of my dissertation is to design and implement an asynchronous Unite and Conquer GMRES/LS-ERAM (UCGLE) based on the unite and conquer approach. UCGLE is proposed to solve large-scale linear systems with the reduction of global communications. This method consists of three computation components: ERAM Component, GMRES Component and LS (Least Squares) Component. GMRES Component is used to solve the systems, LS Component and ERAM Component serve as the preconditioning part. The key feature of this hybrid method is the asynchronous communication among these three components, which reduces the number of overall synchronization points and minimizes the global communication. There are three levels of parallelisms in UCGLE method to explore the hierarchical computing architectures. The convergence acceleration of UCGLE method is similar with a deflated preconditioner. The difference between them is that the improvement of the former one is intrinsic to the methods. It means that in the deflated preconditioning methods, for each time of preconditioning, the solving procedure should stop and wait for the temporary preconditioning procedure. Asynchronous communication of the latter can cover the synchronous communication overhead. Obviously, the asynchronous communication among the different computation components improves the fault tolerance and the reusability of this method. The three computation components work independently from each other, when errors occur inside of ERAM Component, GMRES Component or LS Component, UCGLE can continue to work as a normal restarted GMRES method to solve the problems. In fact, the materials for accelerating the convergence are the eigenvalues. With the help of asynchronous communication, we can select to save the computed eigenvalues by ERAM method into a local file and reuse it for the other solving procedures with the same matrix, which can improve the reusability of linear solvers.

Moreover, both the mathematical model and the implementation of UCGLE are extended to solve a series of linear systems in sequence which share the same operator matrix $A$ but have different Right-hand sides (RHSs) $b$ in sequence. The eigenvalues obtained in solving previous linear systems by UCGLE can be recycled, improved on the fly and applied to construct a new initial guess vector for subsequent linear systems, which can achieve a continuous acceleration to solve linear systems in sequence. Numerical experiments using different test matrices to solve sequences of linear systems on supercomputers indicate a substantial decrease in both computation time and iteration steps when the approximate eigenvalues are recycled to generate the initial guess vectors.

Afterward, since many problems in the field of science and engineering often require to solve simultaneously large-scale non-Hermitian sparse linear systems with multiple RHSs, we develop an extension of UCGLE by combining it with Block GMRES method to solve non-Hermitian linear systems with multiple RHSs. This variant of UCGLE is implemented with novel components and manager engine. This novel engine is capable of allocating multiple Block GMRES at the same time, each Block GMRES solving the linear systems with a subset of RHSs and accelerating the convergence using the eigenvalues approximated by other eigensolvers. Dividing the entire linear system with multiple RHSs into subsets and solving them simultaneously with different allocated linear solvers allow localizing calculations, reducing global communication, and improving parallel performance. Meanwhile, the asynchronous preconditioning using eigenvalues is able to speed up the convergence and improve the fault tolerance and reusability. Numerical experiments using different test matrices on supercomputers indicate that the proposed method achieves a substantial decrease in both computation time and iterative steps with good scaling performance.

Eventually, an adaptive UCGLE is proposed which gives the scheme of auto-tuning the complex parameters inside which have the influence on its numerical and parallel performance. This work was achieved by analyzing the impacts of different parameters on the convergence.

\section{Outline}

The dissertation is organized as follows. In Chapter \ref{State-of-the-art in High-Performance Computing}, we will give the state-of-the-art of HPC, including the modern computing architectures for supercomputers (e.g. CPU, Nvidia GPGPU and Intel Many Integrated Cores) and the parallel programming model (including OpenMP, CUDA, MPI, PGAS, the task/graph based programming, etc.). Finally, in this chapter, we will talk about the current challenges of HPC facing the coming of exascale supercomputers.

Chapter \ref{Krylov Subspace Methods} covers the existing iterative methods for solving linear systems and eigenvalue problems, especially the Krylov Subspace methods. Firstly, this chapter gives a brief introduction of the stationary and non-stationary iterative methods. Then different Krylov Subspace methods will be presented and compared. Apart from the basic introduction of methods, different preconditioners used to accelerate the convergence will be discussed, especially a Least Squares Polynomial method which is used to construct UCGLE, will be introduced in details. The relation between the convergence of Krylov subspace methods for solving linear systems and the spectral distribution of operator matrix $A$ will also be analyzed in this chapter. Finally, we give a bref introduction of the parallel implementation of the Krylov subspace methods on modern distributed memory systems, then discuss the challenges of iterative methods facing on the coming of exascale platforms, and then summerize the recent efforts to adapt the numerical methods to the much larger clusters/supercomputers.

In Chapter \ref{Sparse Matrix Generator with Given Spectra}, we present the parallel implementation and numerical performance evaluation of SMG2S.  SMG2S is able to generate large-scale non-Hermitian test matrices using the user-defined spectrum and ensuring their eigenvalues as the given ones with high accuracy. It is implemented on CPUs and multi-GPU platforms with specific optimized communications. Good strong and weak scaling performance is obtained on several supercomputers. We also propose a method to verify its ability to guarantee the given spectra base on the shift inverse power method. SMG2S is a released open source software which is developed using MPI and C++11. In this chapter, we will analyze the parallel and numerical performance of SMG2S on several supercomputers. Finally, the packaging, the interfaces to other programming laguanges and scientific softwares, and the graphic user interface for verification are introduced in this chapter.

In Chapter \ref{Unite and Conquer GMRES/LS-ERAM Method}, the implementation of UCGLE based on the scientific libraries PETSc and SLEPc for both CPUs and GPUs are presented. In this chapter, we describe the implementation of components, the manager engine, and the distributed and parallel asynchronous communications. After the implementation, the selected parameters, the convergence, scalability and fault tolerance are evaluated on several supercomputers.

Chapter \ref{UCGLE for Linear Systems with Sequences of Right-hand-sides} presents the extension of UCGLE to solve linear systems in sequence with different RHSs by recycling the eigenvalues. Firstly, we give a survey of existing algorithms, including the seed and deflated methods, and then develop the mathematical model and manager engine of UCGLE to solve linear systems in sequence. The experimental results of the evaluation on different supercomputers are also shown in this chapter.

The variant of UCGLE to solve simultaneously linear systems with multiple RHSs is presented in Chapter \ref{UCGLE for Linear Systems with Multiple Right-hand-sides}. Firstly, this chapter introduces the existing block methods to solve linear systems with multiple RHSs, and then analyzes the limitations of block methods on large-scale platforms. Then mathematical extension of Least Squares polynomial for multiple RHSs is given, and then the implementation of the special novel manager engine using MPI Spawn is presented. This engine allows allocating multiple GMRES and ERAM Components at the same time. This special version of UCGLE is implemented with the block GMRES provided by the Belos package of Trilinos. Finally, we give the experimental results on large-scale machines.

UCGLE is a hybrid method with the combination of three different numerical methods. Thus the autotuning of different parameters is very important. In Chapter \ref{Parameters Autotuning}, we propose the strategy of autotuning for several parameters. \textcolor{red}{(to be developed)}

All unite and conquer methods including UCGLE, are able to implement using the workflow/task based environments to manager the fault tolerance, load balance, asynchronous communiactins of signals, arrays and vectors and the management of different computing units such as GPUs. In Chapter \ref{YML and XMP Multi-level Parallelism Programming Paradigm}, we give a glance at the YML framework, and then analyze the workflow of UCGLE methods and the limitations of YML for UC approach. Then we propose the solutions of Grammar and implementation for YML, including the dynamic graphic grammar, the mechanism of exiting parallel branch, and the check-pointing mechanism to manager the fault tolerance of applications.

In Chapter \ref{Conclusion and Pespectives}, we summarize the key results obtained in this thesis and present our concluding remarks. Finally, we suggest some possible paths to future research.
