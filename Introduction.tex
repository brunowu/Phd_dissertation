The exascale era of HPC is coming, and the first real exascale machine will be available around 2020. Under this background,  the HPC cluster systems continue not only to scale up in compute node and Central Processing Unit (CPU) core count, but also the increase of components heterogeneity with the introduction of the Graphics Processing Unit (GPU) and other accelerators. This trend causes the transition to multi- and many cores inside of computing nodes which communicate explicitly through fast interconnection networks. These hierarchical supercomputers can be seen as the combination of distributed and parallel computing. Nevertheless, only a small number of applications may attain sustainable performances due to their lack of good scalability on large clusters. When talking about using the collection of Krylov subspace methods, such as the Generalized minimal residual method (GMRES), the Conjugate Gradient (CG)  and the Biconjugate Gradient Stabilized Method (BiCGSTAB), to solve different kinds of linear systems, with the increase of computing unit number, the communication of overall reduction and global synchronization of applications are a bottleneck.

In details, when solving a large-scale problem on parallel architectures with preconditioned Krylov methods, the cost per iteration of the method becomes the most significant concern, typically because of communication and synchronization overhead. Consequently, large scalar products, overall synchronization, and other operations involving communication among all cores have to be optimized. The numerical applications should be optimized for more local communication and less global communication. To benefit the full computational power of such hierarchical systems, it is central to explore novel parallel methods and models for the solving of linear systems. These methods should not only accelerate the convergence but also have the abilities to adapt to multi-grain, multi-level memory, to improve the fault tolerance, reduce synchronization and promote asynchronization.

The subject of my thesis fits within this research context and it concerns on the smart hybrid Krylov methods for exascale computing based on the unite and conquer approach proposed by Emad \cite{emad2016unite}. This approach is a model for the design of numerical methods by combining different computation components together to work for the same objective, with asynchronous communication among them. Unite implies the combination of different calculation components, and conquer represents different components work together to solve one problem. In the unite and conquer methods, different computation parallel components work independently with asynchronous communication. These different components can be deployed on different platforms such as P2P, cloud and the supercomputer systems, or on the same platform with different processors. The idea of unite and conquer approach came from the article of Saad \cite{saad1984chebyshev} in 1984, where he suggested using Chebyshev polynomial to accelerate the convergence of Explicitly Restarted Arnoldi Method (ERAM) to solve eigenvalue problems. Brezinski \cite{brezinski1994hybrid} proposed in 1994 an approach for solving a system of linear equations which takes a combination of two arbitrary approximate solutions of two methods. In 1999, Essai \cite{essai1999heterogeneous} presented firstly a hybrid gmres/ls-arnoldi method. In 2005, Emad \cite{emad2005multiple} proposed a hybrid approach based on a combination of multiple ERAMs, which showed significant improvement in solving different eigenvalue problems. In 2016, Fender \cite{fender2016leveraging} studied a variant of multiple IRAMs and generated multiple subspaces in a nested fashion in order to dynamically pick the best one inside each restart cycle.

Before the investigation the iterative methods to solve linear systems, we have developped a Scalable Matrix Generator with Given Spectrum (SMG2S), due to the importance of spectral distribution on the convergence of iterative methods. SMG2S allows generating very large dimension non-Hermitian matrices with user-customized eigenvalues. Iterative linear algebra methods are the important parts of the overall computing time of applications in various fields since decades. Recent research related to social networking, big data, machine learning and artificial intelligence has increased the necessity for non-hermitian solvers associated with much larger sparse matrices and graphs. The analysis of the iterative method behaviors for such problems is complex, and it is necessary to evaluate their convergence to solve extremely large non-Hermitian eigenvalue and linear problems on parallel and/or distributed machines. This convergence depends on the properties of spectra. Then, it is necessary to generate large matrices with known spectra to benchmark the methods. These matrices should be non-Hermitian and non-trivial, with very high dimension.

In order to reduce the global communication of iterative methods for solving large systems, we introduce an asynchronous Unite and Conquer GMRES/LS-ERAM (UCGLE) based on the unite and conquer approach. This method composes three computation components: ERAM Component, GMRES Component and LS (Least Squares) Component. GMRES Component is used to solve the systems, LS Component and ERAM Component serve as the preconditioning part. The key feature of this hybrid method is the asynchronous communication among these three components, which reduces the number of overall synchronization points and minimizes the global communication. There are three levels of parallelisms in UCGLE method to explore the hierarchical computing architectures. The convergence acceleration of UCGLE method is similar with a deflated preconditioner. The difference between them is that the improvement of the former one is intrinsic to the methods. It means that in the deflated preconditioning methods, for each time of preconditioning, the solving procedure should stop and wait for the temporary preconditioning procedure. Asynchronous communication of the latter can cover the synchronous communication overhead. Obviously, the asynchronous communication among the different computation components improves the fault tolerance and the reusability of this method. The three computation components work independently from each other, when errors occur inside of ERAM Component, GMRES Component or LS Component, UCGLE can continue to work as a normal restarted GMRES method to solve the problems. In fact, the materials for accelerating the convergence are the eigenvalues. With the help of asynchronous communication, we can select to save the computed eigenvalues by ERAM method into a local file and reuse it for the other solving procedures with the same matrix. 

The dissertation is organized as follows. In Chapter 2, we will give the state-of-the-art of HPC, including the modern computing architectures for supercomputers (such as CPU, Nvidia GPGPU and Intel Many Integrated Cores, etc.) and the parallel programming model (such as OpenMP, CUDA, MPI, PGAS, the task/graph based programming, etc.). Finally, in this chapter, we will talk about the current challenges of exascale computing.

Chapter 3 will cover the existing iterative methods for solving linear systems, especially the Krylov Subspace methods. Firstly, it will give a brief introduction of the stationary and non-stationary iterative methods. Then different Krylov Subspace methods will be presented and compared. Apart from the basic introduction of methods, different preconditioners used to accelerate the convergence will be talked about in this chapter, especially a Least Squares Polynomial methods will be shown in details. The relation between the convergence for solving and the spectral distribution will be also analyzed in this chapter. In the end, we will give the challenges of modern iterative methods facing on the coming of much larger clusters/supercomputers.

In Chapter 4, I will present the parallel implementation and numerical performance evaluation of SMG2S.  SMG2S is able to generate large-scale non-Hermitian test matrices using the user-defined spectrum and ensuring their eigenvalues as the given ones with high accuracy. This generator is implemented on CPUs and multi-GPU platforms. Good strong and weak scaling performance is obtained on several supercomputers. We also propose a method to verify its ability to guarantee the given spectra base on the shift inverse power method. SMG2S is a released open source software which is developed using MPI and C++11. In this chapter, we will analyze the parallel and numerical performance of SMG2S on several supercomputers.

In Chapter 5, the implementation of UCGLE method based on the scientific libraries PETSc and SLEPc for both CPUs and GPUs will be presented. The experimental results for the convergence, parameters, scalability and fault tolerance evaluation will be shown on several supercomputers.

UCGLE can be applied to solving linear systems with multiple or sequent right-hand-sides. In Chapter 6, firstly, we will give a survey of existing methods, including the seed methods, the hybrid methods, the deflated methods and the block methods, etc. Then the implementation of the special manager engine using MPI Spawn for linear systems with multiple right-hand sides will be presented. This engine allows allocating multiple GMRES and ERAM Components at the same time. This special version of UCGLE is implemented with the block GMRES provided by the Belos package of Trilinos. Finally, we give the experimental results on large-scale machines.

UCGLE is a hybrid method with the combination of three different numerical methods. Thus the autotuning of different parameters is very important. In Chapter 7, we propose the strategy of autotuning for several parameters. (to be developed)

In Chapter 8, we will talk about the potential implementation of Unite and Conquer methods with YML graph based languages. 

The final chapter will give the conclusion and perspectives of my thesis research.