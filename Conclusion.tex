\section{Conclusion}

The contributions of this thesis address several interconnected problems in the fields of HPC and the numerical iterative methods for linear systems and eigenvalue problems. This dissertation focuses on the proposition and analysis a distributed and parallel programming paradigm for smart hybrid Krylov methods targetting at the exascale computing.

In the first part, we gave the state-of-the-art of HPC, including the modern computing architectures for supercomputers and the parallel programming models. We also discussed the current challenges of HPC facing the coming of exascale supercomputers.

In the second part, we discussed Krylov iterative methods for the linear systems and eigenvalues problems. We introduced the algorithms of these methods and then analyze the relation between their convergence performance and the spectral information of the operator matrix. We presented different preconditioning techniques to accelerate the convergence of restarted iterative methods, including the preconditioning by matrix, deflation and a selected polynomial.  We explained how to implement the iterative methods in parallel on distributed memory supercomputers. We identified also the correlated goals for the research of parallel implementation of numerical methods facing the upcoming of exascale computing.

In the third part, the parallel implementation and evaluation of SMG2S were given. SMG2S can generate large-scale non-Hermitian test matrices using the user-defined spectrum and ensuring their eigenvalues as the given ones with high accuracy. SMG2S was implemented in parallel both on homogeneous and heterogeneous platforms with good scaling performance. SMG2S is released as an open source software to benchmark the numerical and parallel performance of iterative methods. The proposition of SMG2S was an essential factor for helping to continue my thesis on the evaluation of Krylov methods.

In the fourth part, we supplied the initial implementation UCGLE and its manager engine based on the scientific libraries PETSc and SLEPc for both CPUs and GPUs. We described the implementation of components, the manager engine, and the distributed and parallel asynchronous communications. The selected parameters, the convergence, scalability and fault tolerance are evaluated on several supercomputers. We analyze the impacts of spectral distributions on the convergence of UCGLE by different test matrices generated by SMG2S. UCGLE method was proved to be a good candidate for large-scale computing systems because of its asynchronous communication scheme, its multi-level parallelism, its reusability and fault tolerance, and its potential load balancing. The multi-level parallelism of UCGLE can be flexibly mapped to large-scale distributed hierarchical platforms.

In the fifth part of this dissertation, we have extended UCGLE to solve a series of linear systems in sequence with different RHSs, and showed how to improve the acceleration for solving subsequent linear systems by recycling the dominant eigenvalues. This extension of UCGLE recycles the eigenvalues obtained in solving previous systems, improves them on the fly and constructs a new initial guess vector for subsequent linear systems. Numerical experiments using different test matrices indicate a substantial decrease in both computation time and iteration steps.

In the sixth part, we revisited the implementation UCGLE and proposed a new variant to solve simultaneously linear systems with multiple RHSs, that is $m$-UCGLE. It was implemented with a newly designed manager engine, which can allocate various GMRES components at the same time. For GMRES Component was implemented with block GMRES algorithm, which was intended to solve the linear systems with a subset of RHSs. A mathematical extension of Least Squares polynomial was also proposed for the linear systems with multiple RHSs. $m$-UCGLE was implemented with the packages Belos and Anasazi packages of Trilinos. Its implementation on multi-GPUs was boosted by Kokkos. The experiments on supercomputers proved its good numerical and parallel performance.

In the seventh part, we proposed an adaptive scheme for the selection of the degree of Least Squares polynomial, which is the most critical parameter inside UCGLE.

In the last part, we investigated the possibility of UCGLE using the workflow programming environment YML. Two limitations of the acutal version of YML are found; 1) it does not support the special asynchronous communications between different computation tasks; 2) the lack of a mechanism to check the convergence of iterative methods. We propose the solution by adding a variable to handle the dynamic graphs and optimize the scheduler rules to manage different modes of exiting parallel branches. 

\section{Future Work}

The initial goals of the thesis have been completed, but there are still a lot of problems that should be further explored for the current work.

Since the GPU version of SMG2S was implemented with the data structures and functions provided by PETSc, a specific optimized version for GPUs could be implemented based on CUDA and packaged into the open source software SMG2S in the future.

With the help of SMG2S, the relationship between different existing hybrid and deflated iterative methods and preconditioners for solving non-Hermitian linear systems and the spectral information of operator matrices can be investigated. This work will guide users to select suitable iterative methods according to their applications from the real world.

For UCGLE, a new version should be developed for the case that some eigenvalues of the operator matrix have the positive real part, and the others have the negative part. The possible solution is to construct to two Least Squares polynomials using two polygons built by the eigenvalues either with the all positive or negative real part, separately. This modification can exclude the origin point and might accelerate the convergence of this kind of spectral distribution.

An adaptive UCGLE should be developed, which can autotune in runtime all the complex parameters inside UCGLE. YML, including its grammar and the scheduler policies, should be re-developed the supports for the implementation of the UC approach, then the performance of UCGLE implemented based on workflow can be evaluated. An autotuning scheme can be provided for adaptive selections all the complex parameters of UCGLE.

Finally, the implementation of UCGLE with distributed and parallel programming paradigm is only the beginning of a long adventure. More deflated or hybrid methods can be transformed into the UC scheme, and implemented with a distributed and parallel manner. 


\clearemptydoublepage

\backmatter

\clearemptydoublepage